\documentclass[a4paper, 12pt]{article}
\usepackage[brazil]{babel}
\usepackage[utf8]{inputenc}
%\setlength{\textwidth}{16cm}
\setlength{\textheight}{20cm}
%\hoffset -1cm
\voffset -1cm
\linespread{1.5}

\begin{document}

\begin{center}
\textbf{\large RESUMO}
\end{center}

\vspace{0.5cm}
\indent Como se vê na atualidade, nos métodos de análise de dados de uma determinada área ou indústria, onde há vários equipamentos e grandezas que precisam 
ser monitoradas constantemente, ver-se um número de sensores espalhados pelo ambiente e um tipo de central de aquisição e tratamento destes dados. Este equipamento 
na maioria das vezes é um computador ou equivalente com uma \textit{interface} que integra o usuário/administrador com a rede de sensores. Logicamente, esta solução
é disposta através de um \textit{software} de gerenciamento que é dado pelo fornecedor do equipamento. O \textit{software} do equipamento, que é, no custo todo, 
o que sai caro na maioria das vezes devido ao nível do recurso intelectual investido em tal equipamento. Porêm, alternativas \textit{opensources} têm se mostrado
cada vez mais interessantes em sistemas de automação pois as ferramentas de desenvolvimento são gratuitas e com uma manutenção permanente (dependendo do modelo 
de gerenciamento de projeto, mas a maioria das ferramentas dão suporte ao usuário criando novas atualizações para resolver problemas da versão atual).
\\
\indent Além de diminuir o custo das soluções, outro foco do protótipo é garantir um sistema multiplataforma entre processadores de \textit{32bits}. O conceito 
de multiplataforma para o projeto é tentar gerar uma ferramenta que seja facilmente portável para vários tipos de arquiteturas, garantindo transparência para 
os usuários do protótipo.
\\
\indent O projeto em si, tem como ponto de partida a pesquisa do \textbf{Professor Franklin Martins Pamplona} do \textbf{IFPB} com Grupo de Eletrônica, 
Controle e Automação do IFPB (GECA). Onde utiliza-se uma gama de sensores de corrente e tensão espalhados por uma substação e uma central de aquisição e 
tratamento de dados, conectados através de uma rede \textit{ZigBee} (IEEE 802.15.4). Mais detalhes sobre o projeto em http://www.engeca.org/projetos.

\end{document}

