\documentclass[a4paper, 12pt]{article}
\usepackage[brazil]{babel}
\usepackage[utf8]{inputenc}
\usepackage{graphicx, color}
%\usepackage{fontspec}
%\setmainfont{Arial}
%\setlength{\textwidth}{16cm}
%\setlength{\textheight}{20cm}
\hoffset -1cm
\voffset -1cm
\linespread{1.5}
\begin{document}

\section{INTRODUÇÃO}
\vspace{0.5cm}
\subsection{Origem do Trabalho}

\indent\indent A motivação para a produção deste, foi o intuito de tornar portável um sistema de gerenciamento de sensores, como o confeccionado em (PAMPLONA, 2011),
 escrito em uma linguagem de programação muito utilizada em sistemas operancionais \textit{Linux}. A linguagem utilizada pelo grupo para criar o \textit{software} foi o Python,
 por ser robusta, leve, alto nível e facilmente portável para outras arquiteturas de processadores. 
\\
\indent O porque de escolher uma liguagem de alto nível para utilização no protótipo é outra vantagem, pois se pode configurar o sistema de forma simples
 e rápida, implementando rotinas pequenas sem que o administrador precise de um conhecimento mais denso sobre a máquina. Estes tipos de modelos de sistemas 
podem ser vistos no dia a dia, \textit{Android OS}, que faz com que qualquer usuário com conhecimentos básicos em \textit{java} possa criar suas aplicações ou o \textit{Arduino}, que 
lhe proporciona programar o \textit{firmware} do microcontrolador de \textit{8bits} com uma liguagem bem mais simples que o \textit{C/C++}.

\subsection{Área}
\indent\indent Neste trabalho, foi realizado o desenvolvimento de um conjunto de ferramentas para utilização em \textbf{Instrumentação Eletrônica}. Este conjunto de 
ferramentas está disposto em um sistema operacional \textit{Linux} básico que pode ser reconfigurável facilemente através de uma linguagem de alto nível.
\\
\indent Devido a vantagem de ser multiplataforma, ele pode ser utilizado em vários equipamentos que portam arquitetura de \textit{32bits}. Esta vantagem acaba com o
 esforço de ter que criar um sistema do zero para uma determinada Plataforma de Medição de Parâmetros.
\\
\indent	Através do sistema se pode medir qualquer tipo de variável, seja essa corrente elétrica, sensores de pressão, temperatura, etc. e dependendo do que 
for configurado no sistema, pode-se até atuar na região com motores ou chaves.

\subsection{Problemática}
\indent\indent Nos dias de hoje, as maiores dificuldades na implementação de ferramentas de medição (sensores, rede, \textit{interfaces}, etc) são referentes à 
complexidade em configurar a parte gerencial do sistema, bem como portá-lo para outras plataformas, pois na maioria das vezes o administrador do 
sistema estaria preso a um técnico especializado no \textit{software} de gerenciamento. Isto pode demandar tempo e custar bastante dinheiro.
\\
\indent Os sistemas de gerenciamento atuais, comumente oferecem uma \textit{interface} gráfica que conecta o usuário a rede de sensores espalhadas por determinada 
área como mostra (CAMPOS, 2006) em seu trabalho sobre um sistema de instrumentação para unidades de elevação de petróleo, utilizando o \textit{LabView 8.0} para 
exibir em “alto nível” os dados recebidos pela rede de sensores, não muito diferente do ponto de partida deste trabalho, o detalhe é que uma licença do 
\textit{Labview}  tem um elevado custo e não pode ser moldada de forma específica para o seu sistema. 
\\
\indent Para diminuir os gastos financeiros, teria que ser feita uma análise de relação custo benefício entre a alternativa de se utilizar um \textit{software} 
proprietário ou criar um para determinada aplicação. A segunda alternativa é tão custeosa quanto a primeira, pois haveria de ter um grupo focado em 
desenvolvimento para produzir a ferramenta. Isto requer um pessoal com um conhecimento denso em \textbf{Engenharia de \textit{Software}} e com certeza é bem mais caro do 
que manter uma licença do \textit{LabView}. 
\\
\indent Alternativas \textit{opensource} com foco em gerenciamento fácil e hábil vêm surgindo cada vez mais nos dias atuais como o \textit{ConnectMe} da \textbf{Digi}, que é uma 
plataforma de \textit{32bits} com um sistema operacional \textit{Linux}, dotado de artifícios, como a linguagem de programação \textit{Python}, para escrever as rotinas do sistema 
e acessar os dispositivos externos de forma simples e rápida. Outra alternativa, porêm não \textit{opensource}, é o sistema do roteador 
\textit{Cisco SFS 3504} da \textbf{Cisco}, que inclui tambem, um sistema operacional \textit{Linux} com um \textit{shell script} para configurá-lo. Este modelo de sistemas de gerenciamento 
torna simples e menos caro qualquer implementação de um projeto de sistema de monitoramento. Assim, os gastos com \textit{softwares} \textbf{supervisórios} ou com uma 
dispendiosa equipe de desenvolvedores seriam cortados.

\subsection{Justificativas}
\indent\indent Como mostra (BUSNARDO , 1999) em seu trabalho sobre um sistema de \textit{hardware} e \textit{software} para controle da qualidade de 
energia elétrica, há várias vantagens em aplicar a Instrumentação Eletrônica à processos industriais como:

\begin{itemize}
\item Maior Qualidade: processos controlados de forma automática geram menos erros do que manufaturados;
\item Aumento da Produção: máquinas trabalham mais do que pessoas e oferecem menos perdas;
\item Aumento da Segurança: garantia de estabilidade nos processos industriais;
\end{itemize}

portanto se os métodos de gerenciamento dos processos puderem se dispor de forma clara em um \textit{GUI - Graphical User Interface} (Interface Gráfica
do Usuário), a otimização ou alteração de atividades industrais podem ser feita facilmente. 


\indent A primeira ideia do projeto foi a de manter uma pequena distribuição \textit{Linux} com foco em Automatização de Processos, que pudesse ser 
facilmente configurada, eliminando ao máximo dependência dos administradores à uma equipe técnica. A segunda foi tentar tornar o sistema de gerenciamento 
livre e gratuito, sobre uma licença que fornecesse os devidos direitos de implementação a seus usuários. Entre outras vantagens que este projeto pode proporcionar.

\subsection{Objetivos do Trabalho}



\end{document}
